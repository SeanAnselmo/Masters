% Options for packages loaded elsewhere
\PassOptionsToPackage{unicode}{hyperref}
\PassOptionsToPackage{hyphens}{url}
%
\documentclass[
]{article}
\usepackage{amsmath,amssymb}
\usepackage{lmodern}
\usepackage{iftex}
\ifPDFTeX
  \usepackage[T1]{fontenc}
  \usepackage[utf8]{inputenc}
  \usepackage{textcomp} % provide euro and other symbols
\else % if luatex or xetex
  \usepackage{unicode-math}
  \defaultfontfeatures{Scale=MatchLowercase}
  \defaultfontfeatures[\rmfamily]{Ligatures=TeX,Scale=1}
\fi
% Use upquote if available, for straight quotes in verbatim environments
\IfFileExists{upquote.sty}{\usepackage{upquote}}{}
\IfFileExists{microtype.sty}{% use microtype if available
  \usepackage[]{microtype}
  \UseMicrotypeSet[protrusion]{basicmath} % disable protrusion for tt fonts
}{}
\makeatletter
\@ifundefined{KOMAClassName}{% if non-KOMA class
  \IfFileExists{parskip.sty}{%
    \usepackage{parskip}
  }{% else
    \setlength{\parindent}{0pt}
    \setlength{\parskip}{6pt plus 2pt minus 1pt}}
}{% if KOMA class
  \KOMAoptions{parskip=half}}
\makeatother
\usepackage{xcolor}
\usepackage[margin=1in]{geometry}
\usepackage{color}
\usepackage{fancyvrb}
\newcommand{\VerbBar}{|}
\newcommand{\VERB}{\Verb[commandchars=\\\{\}]}
\DefineVerbatimEnvironment{Highlighting}{Verbatim}{commandchars=\\\{\}}
% Add ',fontsize=\small' for more characters per line
\usepackage{framed}
\definecolor{shadecolor}{RGB}{248,248,248}
\newenvironment{Shaded}{\begin{snugshade}}{\end{snugshade}}
\newcommand{\AlertTok}[1]{\textcolor[rgb]{0.94,0.16,0.16}{#1}}
\newcommand{\AnnotationTok}[1]{\textcolor[rgb]{0.56,0.35,0.01}{\textbf{\textit{#1}}}}
\newcommand{\AttributeTok}[1]{\textcolor[rgb]{0.77,0.63,0.00}{#1}}
\newcommand{\BaseNTok}[1]{\textcolor[rgb]{0.00,0.00,0.81}{#1}}
\newcommand{\BuiltInTok}[1]{#1}
\newcommand{\CharTok}[1]{\textcolor[rgb]{0.31,0.60,0.02}{#1}}
\newcommand{\CommentTok}[1]{\textcolor[rgb]{0.56,0.35,0.01}{\textit{#1}}}
\newcommand{\CommentVarTok}[1]{\textcolor[rgb]{0.56,0.35,0.01}{\textbf{\textit{#1}}}}
\newcommand{\ConstantTok}[1]{\textcolor[rgb]{0.00,0.00,0.00}{#1}}
\newcommand{\ControlFlowTok}[1]{\textcolor[rgb]{0.13,0.29,0.53}{\textbf{#1}}}
\newcommand{\DataTypeTok}[1]{\textcolor[rgb]{0.13,0.29,0.53}{#1}}
\newcommand{\DecValTok}[1]{\textcolor[rgb]{0.00,0.00,0.81}{#1}}
\newcommand{\DocumentationTok}[1]{\textcolor[rgb]{0.56,0.35,0.01}{\textbf{\textit{#1}}}}
\newcommand{\ErrorTok}[1]{\textcolor[rgb]{0.64,0.00,0.00}{\textbf{#1}}}
\newcommand{\ExtensionTok}[1]{#1}
\newcommand{\FloatTok}[1]{\textcolor[rgb]{0.00,0.00,0.81}{#1}}
\newcommand{\FunctionTok}[1]{\textcolor[rgb]{0.00,0.00,0.00}{#1}}
\newcommand{\ImportTok}[1]{#1}
\newcommand{\InformationTok}[1]{\textcolor[rgb]{0.56,0.35,0.01}{\textbf{\textit{#1}}}}
\newcommand{\KeywordTok}[1]{\textcolor[rgb]{0.13,0.29,0.53}{\textbf{#1}}}
\newcommand{\NormalTok}[1]{#1}
\newcommand{\OperatorTok}[1]{\textcolor[rgb]{0.81,0.36,0.00}{\textbf{#1}}}
\newcommand{\OtherTok}[1]{\textcolor[rgb]{0.56,0.35,0.01}{#1}}
\newcommand{\PreprocessorTok}[1]{\textcolor[rgb]{0.56,0.35,0.01}{\textit{#1}}}
\newcommand{\RegionMarkerTok}[1]{#1}
\newcommand{\SpecialCharTok}[1]{\textcolor[rgb]{0.00,0.00,0.00}{#1}}
\newcommand{\SpecialStringTok}[1]{\textcolor[rgb]{0.31,0.60,0.02}{#1}}
\newcommand{\StringTok}[1]{\textcolor[rgb]{0.31,0.60,0.02}{#1}}
\newcommand{\VariableTok}[1]{\textcolor[rgb]{0.00,0.00,0.00}{#1}}
\newcommand{\VerbatimStringTok}[1]{\textcolor[rgb]{0.31,0.60,0.02}{#1}}
\newcommand{\WarningTok}[1]{\textcolor[rgb]{0.56,0.35,0.01}{\textbf{\textit{#1}}}}
\usepackage{graphicx}
\makeatletter
\def\maxwidth{\ifdim\Gin@nat@width>\linewidth\linewidth\else\Gin@nat@width\fi}
\def\maxheight{\ifdim\Gin@nat@height>\textheight\textheight\else\Gin@nat@height\fi}
\makeatother
% Scale images if necessary, so that they will not overflow the page
% margins by default, and it is still possible to overwrite the defaults
% using explicit options in \includegraphics[width, height, ...]{}
\setkeys{Gin}{width=\maxwidth,height=\maxheight,keepaspectratio}
% Set default figure placement to htbp
\makeatletter
\def\fps@figure{htbp}
\makeatother
\setlength{\emergencystretch}{3em} % prevent overfull lines
\providecommand{\tightlist}{%
  \setlength{\itemsep}{0pt}\setlength{\parskip}{0pt}}
\setcounter{secnumdepth}{-\maxdimen} % remove section numbering
\ifLuaTeX
  \usepackage{selnolig}  % disable illegal ligatures
\fi
\IfFileExists{bookmark.sty}{\usepackage{bookmark}}{\usepackage{hyperref}}
\IfFileExists{xurl.sty}{\usepackage{xurl}}{} % add URL line breaks if available
\urlstyle{same} % disable monospaced font for URLs
\hypersetup{
  pdftitle={ASSIGNMENT 1: Multiple Linear Regression},
  hidelinks,
  pdfcreator={LaTeX via pandoc}}

\title{ASSIGNMENT 1: Multiple Linear Regression}
\author{}
\date{\vspace{-2.5em}}

\begin{document}
\maketitle

\hypertarget{first-order-model-with-interaction-term-quantitative-variables}{%
\subsection{First order Model with Interaction Term (Quantitative
Variables)}\label{first-order-model-with-interaction-term-quantitative-variables}}

Deadline: Mar.~8, 2024, by 11:59 pm. Submit to Gradescope.ca

© Thuntida Ngamkham (modified by Danika Lipman) 2024

\textbf{Problem 1}. (From Exercise 1) The amount of water used by the
production facilities of a plant varies. Observations on water usage and
other, possibility related, variables were collected for 250 months. The
data are given in \textbf{water.csv file}. The explanatory variables are

TEMP= average monthly temperature (degree celsius)

PROD=amount of production (in hundreds of cubic)

DAYS=number of operationing day in the month (days)

HOUR=number of hours shut down for maintenance (hours)

The response variable is USAGE=monthly water usage (gallons/minute)

\begin{enumerate}
\def\labelenumi{\alph{enumi}.}
\item
  Fit the model containing all four independent variables. What is the
  estimated multiple regression equation?
\item
  Test the hypothesis for the full model i.e the test of overall
  significance. Use significance level 0.05. Ensure that you include the
  hypotheses, p-value, and conclusion.
\item
  Would you suggest the model in part b for predictive purposes? Which
  model or set of models would you suggest for predictive purposes?
  Hint: Use Individual Coefficients Test (t-test) to find the best
  model. Make sure you provide your hypotheses, p-value(s), and final
  model.)
\item
  Use Partial \(F\) test to confirm that the independent variable
  (removed from part c) should be out of the model at significance level
  0.05. Ensure that you include the hypotheses, p-value, and conclusion.
\item
  Obtain a 95\% confidence interval of regression coefficient for TEMP
  from the model in part c.~Give an interpretation.
\item
  Use the method of Model Fit to calculate \(R^2_{adj}\) and RSE to
  compare the full model in part a and the model in part c.~Which model
  or set of models would you suggest for predictive purpose? For the
  final model, give an interpretation of \(R^2_{adj}\) and RSE.
\item
  Build an interaction model to fit the multiple regression model from
  the model in part f.~From the output, which model would you recommend
  for predictive purposes? Be sure to explain your process.
\end{enumerate}

\begin{Shaded}
\begin{Highlighting}[]
\NormalTok{water }\OtherTok{\textless{}{-}} \FunctionTok{read.csv}\NormalTok{(}\StringTok{"\textasciitilde{}/603/Ass1/water.csv"}\NormalTok{)}
\NormalTok{Q1a }\OtherTok{=} \FunctionTok{lm}\NormalTok{(USAGE}\SpecialCharTok{\textasciitilde{}}\NormalTok{PROD}\SpecialCharTok{+}\NormalTok{TEMP}\SpecialCharTok{+}\NormalTok{HOUR}\SpecialCharTok{+}\NormalTok{DAYS, }\AttributeTok{data=}\NormalTok{water)}
\CommentTok{\#SEAN WRITE IN THE FORMULA HERE Q1a}

\NormalTok{Q1c }\OtherTok{=}  \FunctionTok{lm}\NormalTok{(USAGE}\SpecialCharTok{\textasciitilde{}}\NormalTok{PROD}\SpecialCharTok{+}\NormalTok{TEMP}\SpecialCharTok{+}\NormalTok{HOUR, }\AttributeTok{data=}\NormalTok{water)}\CommentTok{\#removed days because it wasnt significant}
\NormalTok{Q1d }\OtherTok{=} \FunctionTok{anova}\NormalTok{(Q1a, Q1c)}

\NormalTok{Q1e }\OtherTok{=} \FunctionTok{confint}\NormalTok{(Q1c, }\AttributeTok{parm =} \StringTok{\textquotesingle{}TEMP\textquotesingle{}}\NormalTok{)}

\NormalTok{fullRsquared }\OtherTok{=} \FunctionTok{summary}\NormalTok{(Q1a)}\SpecialCharTok{$}\NormalTok{adj.r.squared}
\NormalTok{reducedRsquared }\OtherTok{=} \FunctionTok{summary}\NormalTok{(Q1c)}\SpecialCharTok{$}\NormalTok{adj.r.squared}

\NormalTok{Q1f }\OtherTok{=} \FunctionTok{lm}\NormalTok{(USAGE}\SpecialCharTok{\textasciitilde{}}\NormalTok{(PROD}\SpecialCharTok{+}\NormalTok{TEMP}\SpecialCharTok{+}\NormalTok{HOUR}\SpecialCharTok{+}\NormalTok{DAYS)}\SpecialCharTok{\^{}}\DecValTok{2}\NormalTok{,}\AttributeTok{data=}\NormalTok{water)}
\NormalTok{Q1fReduced }\OtherTok{=} \FunctionTok{lm}\NormalTok{(USAGE}\SpecialCharTok{\textasciitilde{}}\NormalTok{PROD}\SpecialCharTok{+}\NormalTok{TEMP}\SpecialCharTok{+}\NormalTok{HOUR}\SpecialCharTok{+}\NormalTok{PROD}\SpecialCharTok{*}\NormalTok{TEMP}\SpecialCharTok{+}\NormalTok{PROD}\SpecialCharTok{*}\NormalTok{HOUR, }\AttributeTok{data=}\NormalTok{water)}
\end{Highlighting}
\end{Shaded}

\textbf{Answer To Question 1a}

\[
\begin{aligned}
y &= 5.89 + 0.04X_1+ 0.17X_2 - 0.07X_3 - 0.02X_4 \epsilon\\
where\\
y &= \mbox{Usage}\\
X_1 &= \mbox{Prod}\\
X_2 &= \mbox{Temperature}\\
X_3 &= \mbox{Hour}\\
X_4 &= \mbox{Days}\\
\end{aligned}
\]

\textbf{Answer To Question 1b} The null hypothesis is that Temperature,
production, days and hours all do not have an effect on the outcome,
Usage.\[ H_0:PROD/DAYS/HOUR/TEMP = 0\]. This means the model does not
explain any variance in the outcome, Usage. The alternate hypothesis is
\[ H_1:PROD/DAYS/HOUR/TEMP \neq 0\] However, after looking at the p
values of the independent variables it can be concluded that all
variables excluding Days have a significant relationship with Usage (p
values are below 0.05). The p value of the model is 2x10\^{}-16, which
means at least one predictor significantly affects the outcome variable,
Usage.

\textbf{Answer to Question 1c} I would not suggest the model in part B
for predictive purposes, because it includes an independent variable
that does not significantly affect the outcome variable. The p value for
Days is 0.502, which is not below our significance threshold of 0.05. I
would use the model that omits Days, and uses Temperature, Hours, and
production as predictors. The p values for these predictors are all
below 0.05.

\textbf{Prod} = p value is 2x10\^{}-16 \textbf{Temp} = p value is
2x10\^{}-16 \textbf{Hour} = p value is 4.23x10\^{}-5 \textbf{Days} = p
value is 0.502

The final model would look as this:

\[ Usage = \beta_0+\beta_1*Prod+\beta_2*Temp+\beta_3*Hour\]

\textbf{Answer to Question 1d} The anova test shows a test stat of
0.4514, and a p value of 0.5023. This means that we do reject the null
hypothesis in favour of our alternate.

H\_0 = Reduced model and Full model have no significant difference H\_1
= Reduced model has significantly better fit than the full model

Therefore we should drop the Days variable as an indicator.

\textbf{Answer to Question 1e} We are 95\% confident the Temperature
variable has an impact between 0.153, and 0.185. This means that
Temperature usually has a positive relationship with the outcome
variable.

\textbf{Answer to Question 1f} The \(R^2_{adj} = 0.8867\) for our full
model, and the RSE is 1.768. Our reduced model has an
\(R^2_{adj} = 0.8869\) and an RSE of 1.766. The higher adjusted R
squared indicates the model's inputs more accurately describes the
variance. The RSE being slightly lower indicates the reduced model may
be less reliable than the full model.

\textbf{Answer to Question 1g} I would recommend the interaction model
that has interactions between PROD and HOUR, and PROD and TEMP. The
\(R^2_{adj} = 0.9651\) , which is noticeawbly higher than 0.8869 of the
reduced additive model.

\begin{center}\rule{0.5\linewidth}{0.5pt}\end{center}

\textbf{Problem 2}. A collector of antique grandfather clocks sold at
auction believes that the price received for the clocks depends on both
the age of the clocks and the number of bidders at the auction. Thus,
(s)he hypothesizes the first-order model

\[
\begin{aligned}
y &= \beta_0 + \beta_1X_1+\beta_2X_2 +\epsilon\\
where\\
y &= \mbox{Auction price (dollars)}\\
X_1 &= \mbox{Age of clock (years)}\\
X_2 &= \mbox{Number of bidders}
\end{aligned}
\] A sample of 32 auction prices of grandfather clocks, along with their
age and the number of bidders, is given in data file
\textbf{GFCLOCKS.CSV}

\begin{enumerate}
\def\labelenumi{\alph{enumi}.}
\item
  Use the method of least squares to estimate the unknown parameters
  \(\beta_0\), \(\beta_1\), \(\beta_2\) of the model.
\item
  Find the value of SSE that is minimized by the least squares method.
\item
  Estimate \(s\), the standard deviation of the model, and interpret the
  result.
\item
  Find and interpret the adjusted coefficient of determination,
  \(R^2_{Adj}\).
\item
  Construct the ANOVA table for the model and test the global F-test of
  the model at the \(\alpha\) = 0.05 level of significance. Be sure to
  state your hypotheses, p-values, and conclusion.
\item
  Test the hypothesis that the mean auction price of a clock changes as
  the number of bidders increases when age is held constant (i.e., when
  \(\beta_2\neq0\)). (Use \(\alpha\) = 0.05 )
\item
  Find a 95\% confidence interval for \(\beta_1\) and interpret the
  result.
\item
  Test the interaction term between the 2 variables at \(\alpha = .05\).
  What model would you suggest to use for predicting y? Explain.
\end{enumerate}

\begin{Shaded}
\begin{Highlighting}[]
\NormalTok{GFCLOCKS }\OtherTok{\textless{}{-}} \FunctionTok{read.csv}\NormalTok{(}\StringTok{"\textasciitilde{}/603/Ass1/GFCLOCKS.csv"}\NormalTok{)}

\NormalTok{Q2a }\OtherTok{=} \FunctionTok{lm}\NormalTok{(PRICE}\SpecialCharTok{\textasciitilde{}}\NormalTok{AGE}\SpecialCharTok{+}\NormalTok{NUMBIDS, }\AttributeTok{data =}\NormalTok{ GFCLOCKS)}
\NormalTok{Q2b }\OtherTok{=} \FunctionTok{sum}\NormalTok{(Q2a}\SpecialCharTok{$}\NormalTok{residuals}\SpecialCharTok{\^{}}\DecValTok{2}\NormalTok{)}

\CommentTok{\#TO find s we need to solve for sqrt(MSE), where MSE = SSE/df}
\NormalTok{Q2c }\OtherTok{=} \FunctionTok{sqrt}\NormalTok{(Q2b}\SpecialCharTok{/}\DecValTok{29}\NormalTok{)}
\NormalTok{Q2d }\OtherTok{=} \FunctionTok{summary}\NormalTok{(Q2a)}\SpecialCharTok{$}\NormalTok{adj.r.squared}

\NormalTok{Q2e }\OtherTok{=} \FunctionTok{anova}\NormalTok{(Q2a)}
\NormalTok{Q2f }\OtherTok{=} \FunctionTok{summary}\NormalTok{(Q2a)}
\NormalTok{Q2g }\OtherTok{=} \FunctionTok{confint}\NormalTok{(Q2a, }\StringTok{"AGE"}\NormalTok{)}
\NormalTok{Q2h }\OtherTok{=} \FunctionTok{lm}\NormalTok{(PRICE}\SpecialCharTok{\textasciitilde{}}\NormalTok{AGE}\SpecialCharTok{+}\NormalTok{NUMBIDS}\SpecialCharTok{+}\NormalTok{AGE}\SpecialCharTok{*}\NormalTok{NUMBIDS, }\AttributeTok{data =}\NormalTok{ GFCLOCKS)}
\end{Highlighting}
\end{Shaded}

\textbf{Answer to Question 2a}
\(\beta_0 = -1338.95, \beta_1 = 12.74, \beta_2 = 85.95\). The equation
for this model would be: \[ y = -1338.95+12.74X_1+85.95X_2+\epsilon\]

\textbf{Answer to Question 2b} \ensuremath{5.1672654\times 10^{5}}

\textbf{Answer to Question 2c} 133.4846678 This standard deviation means
the actual values deivate from our prediction by about 133.4847
kilojoules per kilowatt-hour.

\textbf{Answer to Question 2d} Our \(R^2_{Adj} = 0.8849\). This means
our model is explaining the variance in prices well. This means there is
a strong relationship between the predictors and the outcome variable,
Price.

\textbf{Answer to Question 2e} c(1, 1, 29), c(2555224.48296325,
1727838.47713747, 516726.539899282), c(2555224.48296325,
1727838.47713747, 17818.1565482511), c(143.405659055905,
96.9706643029277, NA), c(9.52737156185989e-13, 9.34495315616418e-11, NA)
Our null hypothesis is: H0 = None of the predictors have an impact on
the outcome variable. This means the model does not explain any of the
variance. Our alternate hypothesis is H1 = One or more of the predictors
have an impact on the outcome variable. This means the model is able to
explain some of the variance seen.

\textbf{Answer to Question 2f} The p value for NUMBIDS is
9.34x10\^{}-11, which is below the significance threshold of 0.05.
Therefore we can reject the null hypothesis and conclude that NUMBIDS
has a significant positive impact on the price of the clock, while age
is held constant.

\textbf{Answer to Question 2g} The lower limit for our confidence
interval for Age is 10.89, and the upper limit is 14.59. This means we
are 95\% confident the coefficient for Age is between these values. In
clearer terms, we can say that age has a positive correlation with
price, increasing it between 10.89 and 14.59.

\textbf{Answer to Question 2h} I would use the interactive prediction
model. This is because the adjusted R squared is higher compared to the
additive model, (0.9489 versus 0.8849). The RSE is also lower for the
interactive model (88.91 versus 133.5). The p value for the interaction
term is below 0.05, meaning there is a significant interaction between
Age and NUMBIDS affecting Price.

\begin{center}\rule{0.5\linewidth}{0.5pt}\end{center}

\textbf{Problem 3}. \textbf{Cooling method for gas turbines.} Refer to
the Journal of Engineering for Gas Turbines and Power (January 2005)
study of a high pressure inlet fogging method for a gas turbine engine.
The heat rate (kilojoules per kilowatt per hour) was measured for each
in a sample of 67 gas turbines augmented with high pressure inlet
fogging. In addition, several other variables were measured, including
cycle speed (revolutions per minute), inlet temperature (degree
celsius), exhaust gas temperature (degree Celsius), cycle pressure
ratio, and air mass flow rate (kilograms persecond). The data are saved
in the \textbf{TURBINE.CSV} file.

{[}The first and last five observations for the turbine data are listed
in the table.{]}

\begin{enumerate}
\def\labelenumi{(\alph{enumi})}
\item
  Write a first-order model for heat rate (y) as a function of speed,
  inlet temperature, exhaust temperature, cycle pressure ratio, and air
  flow rate.
\item
  Test the overall significance of the model using \(\alpha=0.01\) Be
  sure to state your hypotheses, p-values, and conclusion.
\item
  Fit the best additive model to the data using the method of least
  squares. Test significance of predictors at \(\alpha = 0.06\).Be sure
  to state your hypotheses, p-values, and conclusion.
\item
  Test all possible interaction terms for the best model in part (c) at
  \(\alpha = .06\). What is the final model would you suggest to use for
  predicting y? Explain.
\item
  Give practical interpretations of the \(\beta_i\) estimates.
\item
  Find RSE, \(s\) from the model in part (d)
\item
  Find the adjusted-R2 value from the model in part (d) and interpret
  it.
\item
  Predict a heat rate (y) when a cycle of speed = 273,145 revolutions
  per minute, inlet temperature= 1240 degree celsius, exhaust
  temperature=920 degree celsius, cycle pressure ratio=10 kilograms
  persecond, and air flow rate=25 kilograms persecond.
\end{enumerate}

\begin{Shaded}
\begin{Highlighting}[]
\NormalTok{TURBINE }\OtherTok{\textless{}{-}} \FunctionTok{read.csv}\NormalTok{(}\StringTok{"\textasciitilde{}/603/Ass1/TURBINE.csv"}\NormalTok{)}
\NormalTok{Q3a }\OtherTok{=} \FunctionTok{lm}\NormalTok{(HEATRATE}\SpecialCharTok{\textasciitilde{}}\NormalTok{RPM}\SpecialCharTok{+}\NormalTok{INLET.TEMP}\SpecialCharTok{+}\NormalTok{EXH.TEMP}\SpecialCharTok{+}\NormalTok{CPRATIO}\SpecialCharTok{+}\NormalTok{AIRFLOW, }\AttributeTok{data=}\NormalTok{TURBINE)}
\CommentTok{\#B is CPRATIO and AIRFLOW not sig}
\NormalTok{Q3c }\OtherTok{=} \FunctionTok{lm}\NormalTok{(HEATRATE}\SpecialCharTok{\textasciitilde{}}\NormalTok{RPM}\SpecialCharTok{+}\NormalTok{INLET.TEMP}\SpecialCharTok{+}\NormalTok{EXH.TEMP, }\AttributeTok{data=}\NormalTok{TURBINE)}

\NormalTok{Q3d }\OtherTok{=} \FunctionTok{lm}\NormalTok{(HEATRATE}\SpecialCharTok{\textasciitilde{}}\NormalTok{(RPM}\SpecialCharTok{+}\NormalTok{INLET.TEMP}\SpecialCharTok{+}\NormalTok{EXH.TEMP)}\SpecialCharTok{\^{}}\DecValTok{2}\NormalTok{, }\AttributeTok{data=}\NormalTok{TURBINE)}
\NormalTok{Q3d2 }\OtherTok{=} \FunctionTok{lm}\NormalTok{(HEATRATE}\SpecialCharTok{\textasciitilde{}}\NormalTok{(RPM}\SpecialCharTok{+}\NormalTok{INLET.TEMP}\SpecialCharTok{+}\NormalTok{EXH.TEMP}\SpecialCharTok{+}\NormalTok{RPM}\SpecialCharTok{*}\NormalTok{INLET.TEMP}\SpecialCharTok{+}\NormalTok{RPM}\SpecialCharTok{*}\NormalTok{EXH.TEMP), }\AttributeTok{data=}\NormalTok{TURBINE)}
\end{Highlighting}
\end{Shaded}

\textbf{Answer to Q3a} \$\$

\begin{aligned}
y &= 13,610 + 0.088X_1- 9.2X_2 + 14.39X_3 + 0.35X_4 - 0.85X_5 \epsilon\\
where\\
y &= \mbox{Heat Rate (dollars)}\\
X_1 &= \mbox{RPM (years)}\\
X_2 &= \mbox{Inlet Temperature}\\
X_3 &= \mbox{Exhaust Temperature}\\
X_4 &= \mbox{Cycle Pressure Ratio}\\
X_5 &= \mbox{Airflow}\\
\end{aligned}

\$\$

\textbf{Answer to Q3b} The null hypothesis is
\[ H_0:RPM/InletTemp/ExhuastTemp/CPRatio/Airflow = 0\]. This means the
predictors have no influence on the outcome variables. The alternate
hypothesis is \[ H_0:RPM/InletTemp/ExhuastTemp/CPRatio/Airflow \neq 0\].
This means one or more predictor has an influence on the outcome,
Heatrate. The pvalues of all variables except for CPRatio and Airflow
are below the threshold of 0.01.

\textbf{RPM} = p value of 2.64x10\^{}-8

\textbf{InletTemp}= p value of 6.86x10\^{}-8

\textbf{ExhaustTemp} = p value of 0.000102

\textbf{CPRatio} = p value of 0.9905

\textbf{Airflow}= p value of 0.0598

These p values mean that we are able to say Airflow and CPRatio do not
significantly contribute to the model, and should not be used as
predictors.

This means that CRPratio and Airflow do not contribute to the model and
do not serve predictive purposes. We are also able to reject our null
hypothesis in favor of our alternate, since one or more predictors had a
significant impact on the outcome, Heatrate.

\textbf{Answer to Q3c} Our null hypothesis is
\[ H_0:RPM/InletTemp/ExhuastTemp= 0\] The equation of the best additive
model is: \$\$

\begin{aligned}
y &= 14,360 + RPMX_1+ InletTempX_2 - ExhaustTempX_3  \epsilon\\
\end{aligned}

\$\$ The p values for the best additive model were:

\textbf{RPM} = 2.55x10\^{}-14

\textbf{InletTemperature} = 2x10\^{}-16

\textbf{ExhaustTemperature} = 1.06x10\^{}-7

This means that all of these terms have a significant impact on the
Heatrate. We can therefore reject our null hypothesis in favor of our
alternate hypothesis.

\textbf{Answer to Q3d} The final interactive model equation is:

\[ Heatrate = \beta_0+\beta_1*RPM+\beta_2*InletTemp+\beta_3*Exh\]

\textbf{Answer to Q3e}

\textbf{Answer to Q3f}

\textbf{Answer to Q3g}

\textbf{Answer to Q3h}

\end{document}
